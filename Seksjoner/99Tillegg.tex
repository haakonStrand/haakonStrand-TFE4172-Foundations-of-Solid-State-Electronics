%Tillegg. Flere tillegg legges til ved å lage flere sections:-----------------
\appendix

\section{Kvantemekanisk Notasjon}
\label{notasjon}

Kvantemekanikken introduserer notasjon som kan være utfordrende, appendixen forsøker å gi en oversikt over de ulike notasjonene og deres bruksområder.

\begin{equation*}
    \bra{} -> \text{'bra'}
\end{equation*}

\begin{equation*}
    \ket{} -> \text{'ket'}
\end{equation*}

\begin{equation*}
    \bra{n} -> \text{'bra vektor'} = \psi_n^*(x)
\end{equation*}

\begin{equation*}
    \ket{n} -> \text{'ket vektor'} = \psi_n(x)
\end{equation*}

\begin{equation*}
    \bra{m}\ket{n} = \int_{-\infty}^{\infty}\psi_m^*\psi_n \quad \text{'overlapp integral'}
\end{equation*}

\begin{equation*}
    \bra{m}\hat{A}\ket{n} = \int_{-\infty}^{\infty}\psi_m^*\hat{A}\psi_n \quad \text{'matrise element'}
\end{equation*}

\begin{equation*}
    \bra{n}\ket{n} = \int_{-\infty}^{\infty}\psi_n^*\psi_n \quad \text{= 1 for normalisert }\psi
\end{equation*}

\begin{equation*}
    \bra{n}\hat{A}\ket{n} = \int_{-\infty}^{\infty}\psi_n^*\hat{A}\psi_n = <\hat{A}> \quad \text{Forventningsvedien til }\hat{A}
\end{equation*}

\begin{equation*}
<\hat{A}> = \frac{\bra{n}\hat{A}\ket{n}}{\bra{n}\ket{n}} \text{Forventningsverdi med normalisering}
\end{equation*}

